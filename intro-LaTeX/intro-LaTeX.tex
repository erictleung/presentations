\documentclass{beamer}

\usetheme{Szeged}

\title{Introduction to \LaTeX}
\author{Eric Leung}

\begin{document}

\frame{\titlepage}

\begin{frame}
\frametitle{Overview}

\begin{itemize}
\item Document structure
\item Comments
\item Text formatting
\item Equations
\item Tables
\item Importing Graphics
\item Bibliographies
\end{itemize}

\end{frame}

\section{Document Structure}

\subsection{Minimal Example}

\begin{frame}[fragile]
\frametitle{Minimal Document Example}

\begin{verbatim}
\documentclass{article}

\begin{document}
Hello World.
\end{document}
\end{verbatim}

\end{frame}
\note{Here we have the ``Hello, world!'' of \LaTeX. In other words, this is the
mimimum number of commands to make a \LaTeX~document. What makes it minimal is
the \verb+\documentclass{article}+ and the begin and end tags,
\verb+\begin{document}+ and \verb+\end{document}+, respectively. Although you
cannot see it here, spaces in between words does not matter.}

\subsection{Sections and Subsections}

\begin{frame}
\frametitle{Sections and Subsections}

\end{frame}

\end{document}
